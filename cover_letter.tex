%!TEX TS-program = xelatex
\documentclass[11pt]{friggeri-cover-letter}
\usepackage[brazil,english]{babel}
\usepackage{xltxtra}
\defaultfontfeatures{Ligatures=TeX}
\usepackage{afterpage}
\usepackage{hyperref}
\usepackage{hyphenat}
\usepackage{color}
\usepackage{xcolor}
\usepackage{fancyhdr}
\usepackage{multilanguage}

%\setdoclang{br}{brazil}
\setdoclang{en}{english}

\hypersetup{%
    colorlinks,
    linkcolor={red!50!black},
    citecolor={blue!50!black},
    urlcolor={blue!80!black},
    pdfborder={0 0 0},
}

\pagestyle{fancy}
\lhead{}
\chead{}
\rhead{}
\rfoot{{%
    \footnotesize{%
        \langif{br}{%
        %lang pt-br
            Veja a última versão deste documento em:
            \href{https://github.com/diraol/cv/releases/}{https://github.com/diraol/cv/releases/}%
        }{%
        %lang en
            See the lastest version of this document at:
            \href{https://github.com/diraol/cv/releases/}{https://github.com/diraol/cv/releases/}%
        }
}}}
\renewcommand{\headrulewidth}{0pt}
\renewcommand{\footrulewidth}{0pt}

\hypersetup{%
    pdftitle={Diego Rabatone Oliveira Cover Letter},
    pdfauthor={Diego Rabatone Oliveira},
    pdfsubject={Carta de Apresentação/Cover Letter},
    pdfkeywords={Diego Rabatone Oliveira, diraol},
    colorlinks=false,       % no lik border color
   allbordercolors=white    % white border color for all
}
%\addbibresource{bibliography.bib}
\RequirePackage{xcolor}
\definecolor{pblue}{HTML}{0395DE}

\begin{document}
\thispagestyle{empty}
\pagenumbering{gobble}% Remove page numbers (and reset to 1)
%%%%%%%%%%%%%%%%%%%%%%%%%%%%%%%%%%%%%%%%%%
%\langif{br}{%
%% text for lang pt-br
%
%}{%
%% text for lang pt-br
%
%}
%\langif{br}{}{}
%%%%%%%%%%%%%%%%%%%%%%%%%%%%%%%%%%%%%%%%%%
%header{name}{lastname}{title}{city}{country}{phone}{email}{homepage}
\header{Diego}
       {Rabatone}
       {\langif{br}{Engenheiro de Computação}{Computer Engineer}}
       {São Paulo}
       {\langif{br}{Brasil}{Brazil}}
       {+55~(11)~9~8231~4249}
       {diraol@diraol.eng.br}
       {https://cv.diraol.eng.br}

\textbf{To: Software Carpentry and Data Carpentry}\\
\textbf{About:} Community Development Lead position\\

\vfill

\langif{br}{Caro(a)}{Dear Sir or Madam},

\vfill

my name is Diego Rabatone Oliveira and I am a passionate free software and open
knowledge advocate. I was recommended by Raniere Silva (The University of
Manchester) to apply for this opportunity. I first met Raniere in 2015, when I
participated on a Software Carpentry (SC) workshop that he organized at UNICAMP
(Campinas, Brazil). Since then I have collaborated, as instructor, in two SC
workshops that he organized in São Paulo.

\vfill

My goal to participate on that first SC Workshop was to have knowledge of its
structure, by reason of being involved with courses and workshops organization
forasmuch as 2007. These workshops were the genesis of PoliGNU, a free software
studies group which I co-founded, in 2009, when I was an engineering
undergraduate student. Today I am still a member of PoliGNU, contributing
remotely with web contents, social media and eventually organizing activities,
like the Open Data Day that happened on March 04th, 2017.

\vfill

During my undergraduate studies, I have taught free technologies such as GIMP,
Python, among others, and I have also supported about 10 editions of a LaTeX
course that we developed, summing up more than 500 students from 2009 to 2013.
Later, I got involved with Wikipedia community and help them with some
introductory workshops.

\vfill

For the past few years I have been working mainly as a developer. Firstly on
the Brazilian newspaper O Estado De São Paulo, with data driven journalism.
Then I worked as United Nations for Development Program (UNDP) consultant for
the Brazilian Justice Ministry, developing free software technologies for
social participation. At last but not least, I consulted for Open Knowledge
Foundation, on web data products, working remotely with a fellow from
Philippines and other team members spread around many European countries. As
far as I can, I am always pursuing opportunities with common grounds between
Open Data, Free Software, Open Knowledge and that have a positive social
impact, specially when this solutions are ``people driven''.

\vfill

All in all, I am sure I can contribute with both Software and Data Carpentry
organizations towards its missions, and this opportunity eagers me to face the
challenge to build a Brazilian community around Data Skills, Open Science and
Open Knowledge.

\vfill

Thank you for taking the time to consider my resume.

\vfill

\hfill Yours sincerely,

\hfill \textbf{Diego Rabatone Oliveira}

\vfill

\footnotesize{\thinfont\color{lightgray}\textit{Attached: curriculum vitae - also on:} \textit{\url{http://github.com/diraol/cv/releases}}}

\end{document}
