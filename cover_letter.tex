%!TEX TS-program = xelatex
\documentclass[11pt]{friggeri-cover-letter}
\usepackage[brazil,english]{babel}
\usepackage{xltxtra}
\defaultfontfeatures{Ligatures=TeX}
\usepackage{afterpage}
\usepackage{hyphenat}
\usepackage{color}
\usepackage{xcolor}
\usepackage{hyperref}          % clickable URls and cross-references
\usepackage{fancyhdr}
\usepackage{multilanguage}

%\setdoclang{br}{brazil}
\setdoclang{en}{english}

\pagestyle{fancy}
\lhead{}
\chead{}
\rhead{}
\rfoot{{\footnotesize{\langif{br}{%
}{%
  \footnotesize{\thinfont\color{lightgray}\textit{Attached: curriculum vitae - also on:}
  \textit{\hypersetup{allbordercolors=white}\url{http://github.com/diraol/cv/releases}}}
}}}}
\renewcommand{\headrulewidth}{0pt}
\renewcommand{\footrulewidth}{0pt}
%\addbibresource{bibliography.bib}
\RequirePackage{xcolor}

\hypersetup{%
  pdftitle={Diego Rabatone Oliveira Cover Letter},
  pdfauthor={Diego Rabatone Oliveira},
  pdfsubject={Carta de Apresentação/Cover Letter},
  pdfkeywords={Diego Rabatone Oliveira, diraol},
  % colorlinks,
  linkcolor={red!50!black},
  citecolor={blue!50!black},
  urlcolor={blue!80!black},
  %pdfborder={0 0 0},
  pdfborderstyle={/S/U/W 1}, % underline links instead of boxes
  linkbordercolor=red,       % color of internal links
  citebordercolor=green,     % color of links to bibliography
  filebordercolor=magenta,   % color of file links
  urlbordercolor=cerulean,        % color of external links
  % colorlinks=true,       % no lik border color
  %allbordercolors=white    % white border color for all
}

\begin{document}
\thispagestyle{empty}
\pagenumbering{gobble}% Remove page numbers (and reset to 1)
%%%%%%%%%%%%%%%%%%%%%%%%%%%%%%%%%%%%%%%%%%
%\langif{br}{text for lang pt-br}{text for lang en}
%\langif{br}{}{}
%%%%%%%%%%%%%%%%%%%%%%%%%%%%%%%%%%%%%%%%%%
%header{name}{lastname}{title}{city}{country}{phone}{email}{homepage}
\header{Diego}
       {Rabatone}
       {\langif{br}{Engenheiro de Computação}{Computer Engineer}}
       {São Paulo}
       {\langif{br}{Brasil}{Brazil}}
       {+55~(11)~9~8231~4249}
       {diraol@diraol.eng.br}
       {https://cv.diraol.eng.br}

\textbf{To: Software Carpentry and Data Carpentry}\\
\textbf{About:} Community Development Lead position\\

\vfill

\langif{br}{Caro(a)}{Dear Sir or Madam},

\vfill

My name is Diego Rabatone Oliveira and I am a passionate free software and open
knowledge advocate.
%
I was recommended by Raniere Silva (The University of Manchester) to apply for
this opportunity.
%
I first met Raniere in 2015, when I participated on a Software Carpentry (SC)
\href{https://rgaiacs.github.io/2015-06-04-unicamp/}{workshop} that he
organised at the Universidade de Campinas (UNICAMP).
%
Since then I have collaborated, as instructor, in two SC workshops that he
organised in São Paulo (\href{https://rgaiacs.github.io/2015-09-10-usp/}{2015}
and \href{https://rgaiacs.github.io/2016-05-27-ccsl/}{2016}).

%\vfill

My goal to attend the Workshop at UNICAMP as a learner was to have knowledge of
its structure, by reason of being involved with courses and workshops
organisation forasmuch as 2007.
%
These workshops were the genesis of \href{https://polignu.org}{PoliGNU}, a free
software studies group which I co-founded, in 2009, when I was a computer
engineering undergraduate student at the Universidade de São Paulo, Brazil.
%
As a PoliGNU member, I contributed by producing web contents, managing the
group social media networks and organising lots of activities.

\vfill

The most common activity was to organize workshops related to free
technologies, such as GIMP, Python, among others, and I have also supported
about 10 editions of a LaTeX course that we developed, summing up more than 500
students from 2009 to 2013.
%
By that time, I did not know Software Carpentry, but I am sure that its lessons
would have helped a lot to improve our activities.
%
Also, on 2015 I formed a group that have subscribed to be a local group on a
Software Carpentry Instructor Training.
%
Unfortunately, we were not selected, but I am still interested in participating
on a Software Carpentry Instructor Training on the near future.
%
%Later on, I got involved with Wikipedia community and help them with
%some introductory workshops on my University.

%\vfill

Beyond PoliGNU solo activities, we also established a strong relationship with
the USP FLOSS Competence Center (\href{http://ccsl.ime.usp.br/en}{CCSL-USP}),
being their major partners on the Engineering School (Escola Politécnica).
%
I was the PoliGNU member responsible for this partnership and our joint
activities.
%
One outcome of this cooperation is that we built a project called
\href{http://radarparlamentar.polignu.org}{Parliamentary Radar}, that was four
times awarded, a project that still exists and now have gathered new
collaborations.
%
A CCSL PhD became a Professor at Universidade de Brasília (UNB), and since four
years ago his Software Engineering students collaborate with our project guided
by our mentoring. We have reached more than 50 mentored students since then.

%\vfill

For the past few years, I have been working mainly as a developer.
%
Firstly I was a member of the first Brazilian data driven journalism team
(\href{http://blog.estadaodados.com}{Estadão Dados}), on the Brazilian
newspaper O Estado De São Paulo. There I was the only technical team member,
working alongside with three more journalists. The main learnings from that
period are related to my role on translating the journalists ideas into
products that were easy for the readers to understand, while being able to
delivery those data driven products all by myself.
%
Then I worked as an United Nations for Development Program (UNDP) consultant
for the Brazilian Justice Ministry, working on a project to improve social
participation. There I played an important role on organizing the team of
consultants by guiding the implementation of agile techniques, which made
possible for us to deliver the required products.
%
I also consulted for Open Knowledge Foundation, on web data products, working
remotely with a School of Data Fellow from Philippines, among other team
members spread around many European countries.
%
At last but not least, since last year I am a Python and Data Science Mentor at
\href{https://www.thinkful.com/mentors/}{Thinkful}, which is being an enrichful
experience on understanding the different learning processes that each person
have.

%\vfill

From all my technical and educational experiences, as student, instructor and
courses organiser, I have come across many insights about learning, specially
when related to technological knowledges. I have a strong believe that the more
the students learn by practice, the more they will strengthen the knowledge.
Moreover, it is necessary to make them understand the basic and more general
concepts, the way of thinking on how to code, apart from the language or
technology they are working with. That is the main reason why I like so much
the Problem Based Learning (PBL) methodology, an approach where you start by
practicing and then you try to figure out the concepts from that practical
knowledge. There are two other topics that are necessary to keep in mind while
organizing introductory workshops. The first is to strengthen students
confidence about writing code, by showing them that there is no problem on
making mistakes, and any code will be rewritten many times \- or to be
improved, or to be corrected. The second is that the code the students will
produce need to be useful for them, not only some piece of code that will be
thrown away after the workshop.

%\vfill

In my opinion, the current teaching techniques are still too much attached to
the tradicional lecture based methods, even on technological courses. Since
this is the usual background, for both instructors and students, a special work
need to be done with the instructors, teaching them how to use more modern
approaches, such as PBL. For this to be possible, there is the necessity of a
prior hard work in order to produce guidelines and a solid course structure for
that kind of approach. I am sure I can contribute a lot with that goal, once I
am highly experienced on short period programming events, such as Hackathons,
in addition to my strong technological background.

%\vfill

All in all, I am always pursuing opportunities that have common ground between
Open Data, Free Software, Open Knowledge and that have a positive social
impact, specially when they are people driven. Thus, it will be satisfying to
contribute with both Software and Data Carpentry organisations towards its
missions, and this opportunity eagers me to face the challenge to build a
Global South community around Programming and Data Skills that would help to
improve both Reproducibility and Replicability on academia.

\vfill

Thank you for taking the time to consider my resume.

\vfill

\hfill Yours sincerely,

\hfill \textbf{Diego Rabatone Oliveira}

\vfill

\end{document}
